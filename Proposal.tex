\documentclass[a4paper,12pt]{CSUResearchProposal}
\usepackage[top=2.4cm, bottom=3cm, left=2.5cm, right=2.5cm]{geometry}
\usepackage{indentfirst}
\usepackage{extpfeil}
%\usepackage{times}  % Times fonts
\usepackage{titlesec}
\usepackage{hyperref}
\usepackage{graphicx}
\usepackage{subfigure}
%\usepackage[ruled,vlined,norelsize]{algorithm2e}
\usepackage{esint}
\usepackage{paralist}
\usepackage{enumerate}
\usepackage{amssymb,amsmath}
\usepackage{amsbsy}
\usepackage{amsthm}
\usepackage{type1cm}
\usepackage{type1ec}

%\usepackage[a4,frame,center]{crop}
%\usepackage{noinfo}
% \usepackage{stmaryrd}
%\usepackage[square,sort,comma,numbers]{natbib}
\usepackage{natbib}
%\usepackage{javen}
\usepackage{bbm}


\newcommand{\NP}{\mathcal{NP}}

% \xeCJKsetup{CJKglue=\hspace{0pt plus .08 \baselineskip }}  
%\titleformat{\section}{\Large\bfseries}{\thesection.}{1em}{}


\linespread{1.56}

\usepackage{float}
\usepackage{algorithm}
\usepackage{algorithmicx}
\usepackage{algcompatible}
\floatname{algorithm}{\zihao{-5}算法}
\renewcommand{\algorithmicrequire}{\textbf{输入:}}
\renewcommand{\algorithmicensure}{\textbf{输出:}}
\begin{document}
\author{XXX}
\StudentID{12345678}
\ThesisTitle{开题报告模板}
\SubjectFrom{学校自选题目}
\School{地球科学与信息物理学院}
\Major{地质资源与地质工程~地球物理班}
\Degree{硕士}
\Supervisor{xxx~教授}
\Date{2018年11月6日}

\ReviewerOne{A}{教授}{XXX学院}
\ReviewerTwo{B}{教授}{XXX学院}
\ReviewerThree{C}{教授}{XXX学院}
\ReviewerFour{D}{教授}{XXX学院}
\ReviewerFive{E}{教授}{XXX学院}
\ReviewerSix{F}{教授}{XXX学院}

\Secretary{B}{教授}{XXX学院}
\maketitle



\section{选题意义和研究价值}
\label{sec:background}
参考文献示例\citep{panet2014mapping}.

公式示例
\begin{equation}\label{equ:potential}
V = G\rho \iiint\limits_\Omega  {\frac{1}{{\sqrt {\Delta x_1^2 + \Delta x_2^2 + \Delta x_3^2} }}d\Omega }
\end{equation}

图片示例
\begin{figure}[H]
	\centering
	\includegraphics[width=0.6\linewidth]{figures/gr}
	\caption{图片示例}
	\label{fig:gr}
\end{figure}

\begin{picture}(100,10)(0,0)
\put(-29.5,5){\line(1,0){472.5}}
\end{picture}
  
\section{国内外研究现状和发展动态}
\label{sec:current_state}

\begin{picture}(100,10)(0,0)
\put(-29.5,5){\line(1,0){472.5}}
\end{picture}
\section{主要研究思路、研究内容和在学术方面的创新点}
\label{sec:contents_research}

\begin{picture}(100,10)(0,0)
\put(-29.5,5){\line(1,0){472.5}}
\end{picture}

\section{拟采取的研究方法和技术路线}
\label{sec:method}


\begin{picture}(100,10)(0,0)
\put(-29.5,5){\line(1,0){472.5}}
\end{picture}

\section{进度安排和预期成果}

\begin{picture}(100,10)(0,0)
\put(-29.5,5){\line(1,0){472.5}}
\end{picture}
\section{已有基础{\kaishu (与本项目有关的工作积累和已取得的成绩、已具备的条件、尚缺少的条件及解决途径)}}


\begin{picture}(100,10)(0,0)
\put(-29.5,5){\line(1,0){472.5}}
\end{picture}
\section{主要参考文献}
\vspace*{-3ex}
\bibliographystyle{model5-names.bst}
\bibliography{ref}
%\bibliographystyle{IEEEtranN}   % (uses file "plain.bst")
%\bibliography{ref}      % expects file "myrefs.bib"

\end{document}

%%% Local Variables: 
%%% coding: utf-8
%%% mode: latex
%%% TeX-engine: xetex
%%% End: 
